\chapter[Results and future developments]{Results and future developments}

\section[Results]{Results}


A figure example, with text in line (NO CAPTION)
\begin{center}

\includegraphics[scale=0.15]{head/logo.png}


\end{center}

A figure example, with floating object and caption

\begin{figure}
\centering

\includegraphics[scale=0.15]{head/logo.png}
\caption{the logo of UniTo}

\end{figure}


\section[Further developments]{Further developments}



\section{Conclusions}

During the measurement campaign of this year we learnt how to operate and measure the breakdown rate of the structure with and without beam.

The beam effect analysis has still to be carried on in detail, in particular have to be understood if when running with beam the conditioning takes place or not. Further investigations on this topic require a stable and extensive beam time, which was not the case of the CTF. The comparison of this data with the ones of a stable long test with beam can suggest that switching condition ripetutamente w/ w/o beam can lead to a higher BDR ...
- as DC tests suggest
- as we cannot see because of the low rep.rate
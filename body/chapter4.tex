\chapter[Data analysis]{Data analysis}

\section[Offline selection of the events]{Offline selection of the events}

In order to perform the data analysis, a full analysis framework have been developed using MATLAB. 

\subsection[Data collected from the PXI]{Data collected from the PXI}

As mentioned in the previous chapter, the data acquisition is performed from a real-time LabVIEW program running into the PXI crate. When an event hits at least one of the online triggers is saved into the crate's disk. Every 8 hours, the crate assemble a data file and stores it in TDMS format \cite{NI:TDMS}. To proceed with the analysis, the files are merged joining all the data of the same day together in a new data file. The original data are anyway kept stored. All these actions were found already implemented in the system at the beginning of this work.

The kind of events that are collected by the PXI crate have two natures:
\begin{enumerate}
\item Interlock pulses: an event that triggered one of the online interlock. If available, the two precedent pulses are saved.
\item Backup pulses: if no event triggers an online interlock, a pulse every minute is saved.
\end{enumerate}


\subsection[Data selection for high-gradient tests]{Data selection for high-gradient tests}

Of th





\section[Time and space positioning of the breakdowns]{Time and space positioning of the breakdowns}







\section[Beam induced RF]{Beam induced RF}

%%%%%%%%%%%%%FIX EQUATION NUMBERING !!!


\chapter[Accelerating structures]{Accelerating structures}

The accelerating structures are one of the key part of a particle accelerator. Even though in the first accelerators the particle acceleration was achieved simply with fixed fields, this technique showed its limitations quite soon due to the impossibility to create arbitrary high DC voltages without incur in electrostatic discharges. Nowadays the conventional technology for particle acceleration is based on the RF cavities, so it is natural to push the state-of-the-art technology the further possible in order to reach the highest acceleration possible in the shortest length. In the case of the CLIC project a key issue is to be able to produce cavities with an accelerating field of 100 MV/m, which is the cutting-edge value at the moment when keeping the breakdown rate limited.

In this chapter some topics about accelerating cavities will be presented, with particular attention to the travelling-wave accelerating structures and the limitations of such parts. Further details on different cavities types and their design and working principle are available in \cite{Wangler:RF_LINAC} and \cite{Humphries:107756}.

\section[Travelling wave accelerating structures]{Travelling wave accelerating structures}

\subsection[Reminder of electromagnetism]{Reminder of electromagnetism}

\subsubsection{Maxwell's equations}

It's well known since the basic courses of physics that the propagation of the electromagnetic fields follow the Maxwell's equations, that in a medium are

\begin{center}
\begin{tabular}{ l l r }
$\nabla . \vec{D} = \rho_{free}$			\hspace{10mm}				&	$\nabla . \vec{B} = 0	$	& \hspace{10mm} \multirow{2}{*}{(2.1)}\\
$\nabla \times \vec{E} = - \frac{\partial \vec{B}}{\partial t}$		&	$\nabla \times \vec{H} = \vec{J}_{cond} + \frac{\partial \vec{D}}{\partial t}	$ &\\
\end{tabular}
\end{center}
where $\vec{E}$ and $\vec{B}$ are the electric and magnetic field in the vacuum, $\vec{D} = \epsilon \vec{E}$, $\vec{B} = \mu \vec{H}$ and $\vec{E} = \sigma \vec{J}$. The propagation in vacuum is described by the same equations and can be easily derived with an appropriate choice of the constants.

\subsubsection[Waveguides and cavities]{Waveguides and cavities}

In some particular cases the confined propagation of electromagnetic waves is possible, and is commonly realised using a metal waveguide to direct all the energy in a single direction. In this section some useful results on the propagation in a waveguide will be stressed, the full derivation can be found in \cite{Botta:EM, Jackson:ClassEM}. 

The request of propagation in the direction of the axis of the waveguide can be traduced to the condition $\vec{E} \times \vec{n} = 0$, that means asking that no power is dissipated in the walls of the cavity since the electric field is normal to the surface anywhere and anytime.

Many solutions can be found to the problem, integrating the Maxwell's equations with the given boundary condition, but without entering in the calculations, the most interesting solutions belong to two classes:
\begin{itemize}
\item \textbf{TM modes:} Transverse Magnetic modes, where the axial component of the magnetic field is null
\item \textbf{TE modes:} Transverse Electric modes, where the axial component of the electric field is null
\end{itemize}
where every mode has a particular cutoff frequency, and will propagate a wave only in case that the condition $\omega > \omega_c$ is met (where $\omega_c = 2\pi f_c$ and $f_c$ is the cutoff frequency of the mode). The cutoff frequency depends mainly by the geometry of the guide utilised, and waves with a frequency lower than the cutoff frequency will be exponentially damped.

A particular case is represented by the \textit{Resonant Cavities}, which are closed waveguides (e.g. a cylindrical cavity, or \textit{pillbox cavity}, is composed of a circular waveguide closed by two planes at the extremities). In the resonant cavities the EM field can resonate according to the eigenfrequencies of the





\subsection[Periodic structures and synchronous particle acceleration]{Periodic structures and synchronous particle \\acceleration}

In order to deliver a  constant energy gain to a charged particle two conditions have to be met:

\begin{enumerate}
\item The electromagnetic wave has to have a non-zero component in the direction of the motion of the particle
\item The phase velocity of a wave have to be the same of the particle velocity, in order to keep the relative phase constant, and so the acceleration
\end{enumerate}
using some results of the theory of EM waves it is possible to derive that the former cannot be accomplished in the free space, but can be easily met in a cavity (e.g. a waveguide), and to satisfy the latter is not sufficient simply using a waveguide, but is necessary a particular geometry, typically periodic geometries are used for this purpose.







\cite{Botta:EM}




\subsection[Constant gradient vs constant impedance structures]{Constant gradient vs constant impedance structures}
Lorem ipsum dolor sit amet, consectetur adipiscing elit. Sed dui sem, aliquam id ultricies sit amet, fermentum at magna. Aenean vitae rhoncus leo. Fusce gravida consequat lacus, a porta risus bibendum semper. Morbi eget auctor velit. Pellentesque eu lacinia nisi. Maecenas sed orci eu erat porta imperdiet ac non dui. Pellentesque a odio ac quam euismod tempor. Nulla in dapibus mauris, a sodales ex. In imperdiet enim sed ornare sollicitudin. Pellentesque habitant morbi tristique senectus et netus et malesuada fames ac turpis egestas. Donec vehicula metus eu nisi ornare euismod. Proin at ex non ex iaculis porta.

\subsection[Beam effect on the accelerating structure]{Beam effect on the accelerating structure}
Lorem ipsum dolor sit amet, consectetur adipiscing elit. Sed dui sem, aliquam id ultricies sit amet, fermentum at magna. Aenean vitae rhoncus leo. Fusce gravida consequat lacus, a porta risus bibendum semper. Morbi eget auctor velit. Pellentesque eu lacinia nisi. Maecenas sed orci eu erat porta imperdiet ac non dui. Pellentesque a odio ac quam euismod tempor. Nulla in dapibus mauris, a sodales ex. In imperdiet enim sed ornare sollicitudin. Pellentesque habitant morbi tristique senectus et netus et malesuada fames ac turpis egestas. Donec vehicula metus eu nisi ornare euismod. Proin at ex non ex iaculis porta.


\subsection[Figure of merit for accelerating structures]{Figure of merit for accelerating structures}

R/Q and stuff like that

something about the fact that simulation and not analitical laws are used


\section[High power limits and scaling laws]{High power limits and scaling laws}

The limiting factors for room-temperature high-gradient accelerators have been identified as \textit{field emission} and \textit{RF breakdown}. The former is the emission of electrons in the form of the so called \textit{"Dark current"}, that subtracts RF power, causes radiation and can produce wakefields; the latter is a limiting factor to the operation of accelerators and can damage the structures.\cite{Wang:1997ip}

The understanding of these phenomena is particularly challenging and requires a mixture of notions of disciplines such as surface physics, metallurgy, fabrication processes, microwaves, beam dynamic and plasma physics. At the moment a satisfactory unified theory of the processes that take place during the breakdowns have not been found yet, then the improvement of the structures is achieved using some scaling laws for the high power limitations, that have been deducted from the experience and the experiments on the structures tested so far. 


\subsection[Field emission law]{Field emission law}

\subsubsection{Emission from flat clean surface}

The field emission law was theorised by Fowler and Nordheim in 1928 and rule the current emission from a metal with applied an intense electric field. The derivation was carried on calculating the tunnel probability of electrons of the conduction band through the perfectly flat and clean surface of a metal. \\The applied electric field modify the potential barrier, and the current density of emitted electrons can be derived as the following, giving the  \textit{Fowler-Nordheim equation} \cite{Fowler173}
\begin{equation}
J_F = \frac{ 1.54\times10^{-6} \times 10^{4.52\phi^{-0.5}} E^2}{  \phi } \, \text{exp} \left ( -\frac{6.53\times 10^9 \phi^{1.5}}{E} \right ) \quad [A.m^{-2}]  \label{FNlaw}
\end{equation}
where $\phi$ is the work function of the material and $E$ is the applied electric field.

\subsubsection{Enhanced field emission}

It's well known that almost any surface is never perfectly clean and flat, and also the fact that the asperities of the surface provoke an enhancement of the local electric field. This behaviour lead to the phenomenon known as \textit{Enhanced Field Emission} (EFE), which major contributors are:
\begin{itemize}
\item Surface imperfection due to imperfect machining
\item Metallic dust
\item Molten craters after breakdowns
\item Absorbed gas
\end{itemize} 
and some others. These effects can create particular sites known as "emitters". It's a common praxis define the field enhancement factor $\beta$ to relate the electric field to the microscopic one
\begin{equation}
E_{m} = \beta E
\end{equation}
and the $\beta$ factors can be calculated according to the emitter's geometry \cite{Rohrbach:190223} as exploited in figure \ref{tip_factors}.
Once the local field is known, using the formula \ref{FNlaw} calculate the current emitted from EFE by an emitter site of area A gives 
\begin{equation}
I_F = \frac{ 1.54\times10^{-6} \times 10^{4.52\phi^{-0.5}} A \beta^2 E^2}{  \phi } \, \text{exp} \left ( -\frac{6.53\times 10^9 \phi^{1.5}}{\beta E} \right ) \quad [A]  \label{If}
\end{equation}
where $\beta E$ is the local field, $\phi$ is the work function of the material and $A$ the area of the considered emitter.

 In the RF case the average current emitted is given by similar calculations, averaging the electric field on an RF period. The full calculation is stressed in \cite{Wang:1997ip}. 

Experimental evidence of the dark current emission have been detected by setups equipped with Faraday Cups, as in \cite{Wuensch:advaces}.

The emission of dark current seems to be a precursor of the breakdown  process, even if the relationship between the two processes has not been clarified so far.

\begin{figure}[h]
\centering

\includegraphics[scale=0.3]{pictures/beta_tip_val}
\caption{Field enhancement factors for simple geometries of metallic protrusions, plotted as function of geometrical features. From \cite{Rohrbach:190223}}
\label{tip_factors}

\end{figure}


%%%%%% EXCURSUS ABOUT REASONABLE VALUES OF BETA




\subsection[Kilpatrick's critereon]{Kilpatrick's critereon}

The Kilpatrick's Critereon was the first attempt to create a high power limit valid both in DC and RF applications \cite{KilpLimit}. The model was based on the acknowledgement of the Field Emission, and suggesting that the vacuum arc was created by the cascade of secondary electrons ejected from the surface by ion bombardment.

Using data collected in the 1950's, the following empirical law was formulated
\begin{equation}
W E^2 \text{exp} \left (  -1.7\times 10^5 E^{-1}  \right ) = 1.8 \times 10^{14}
\label{KilpLaw}
\end{equation}
where $W$ is the maximum possible ion energy in $eV$, and $E$ is the field in $V/m$. This can also be rewritten in the case of the RF to give a frequency limit, as
\begin{equation}
f = 1.64 \,\, E^2 \,\, \text{exp} \left (  -\frac{8.5}{E} \right )
\end{equation}
where $f$ is the frequency in $MHz$ and $E$ is the electric field in $MV/m$.

\begin{figure}[h]
\centering

\includegraphics[scale=0.5]{pictures/kilpatrickCrit}
\caption{Kilpatrick's original plot. Note that $MC = MHz$ (\textit{Mega Cycles}) and that $W$ for RF is function of frequency, voltage and gap amplitude.\\ See paper for details  \cite{KilpLimit}}
\label{kilpPlot}

\end{figure}

The critereon was reviewed many times up to now, because the experiments conducted nowadays show a limit up to 10 times lower than Kilpatrick's prediction, and this can be addressed to different reasons: first of all the quality of the machining of the structures have increased considerably since the 1950's; in second instance the formula for $W$ was deducted for parallel plates, but the condition inside the RF cavities are different during operations; and finally the key assumption was that the breakdown was triggered by the secondary emission provoked by the ion bombardment.

%%%%% GOOD OF KILPATRICK --> BD and FIELD EMISSION ARE PROVOKED BY DIFFERENT PHENOMENA

At the end of the 1980's J.D.Wang finally proposed a model based on microprotrusion effect on the field and field emission, that involves the formation of a micro-plasma during the breakdown process. Also the Kilpatrick's limit was revised again in order to match the experimental results. \cite{Wang:1986, kilp:story}






\subsection{The modified Poynting vector $S_c$}

Different other scalings have been proposed so far involving power flow, such as P/C scaling (related to the maximum surface electrical field) or a more advanced one  \cite{Wuensch:1004189}, but even the last one showed some deviations from expectations in part of the structures tested and is applicable just to travelling wave structures (TWS) since in standing wave structures (SWS) the power flow is close to zero.

During the work on high gradient normal conducting accelerating structures for the CLIC project, was however developed a new field quantity, the \textit{Modified Poynting Vector} $S_c$, that is suitable both for TWS and SWS. \cite{Grudiev:newLoc}

The founding of this %%%TERRIBLE

are that: the breakdown process is a determined by the accumulation of the pulses rather than the single pulse and the possible triggers of the breakdown can be induced by many processes that will be discussed later and are not relevant for the scaling law derivation.

A number of effects have to be taken into account (a simple geometry is considered, a cilindrical protrusion surmounted by a emispherical cap): 

\subsubsection{Pulsed heating by field emission current}

it's known that the field emission gets enhanced by the presence of the protrusion as described before, and in this case the field enhancement factor can be expressed as $\beta \simeq h/r$, where $h$ is the tip height and $r$ is the cap radius. So the tip will emit a current according to the Fowler-Nordheim law, causing in first approximation the heating of the tip due to the ohmic heating. Assuming that the edge of the tip will be the most heated part, and using the heat conduction equation, it is possible to derive the emitted current to bring the tip to the melted state, which is found to be approximately $36\, A/\mu m^2$ for a tip of $1 \, \mu m$ height and a pulse of $100\, ns$. This is consistent with the findings in \cite{soviet:1983}. The $\beta$ factor can be then derived, which is approximately between 40 and 60 considering a surface electric field in case of breakdown between 200 and 300 $MV/m$.


\begin{figure}[h]
\centering

\includegraphics[scale=0.3]{pictures/field_S_c}
\caption{Electric field distribution around the protrusion considered. Arrows indicate the direction of the field and the color code the absolute value of the field, mapped logarithmically \cite{Grudiev:newLoc} }
\label{kilpPlot}

\end{figure}



\subsubsection{Power flow near an emission site}

The heating of the tip mentioned before requires a huge amount of energy, which can be supplied only by the RF power present into the cavity. This is described by the Poynting vector $S_{RF} = E\times H_{RF}$. As discussed before a current is established in the tip, and flows through, subtracting energy to the EM field in the surroundings of the tip; the current flows through the tip and leave it at the edge, getting sprayed in the cavity according to the Fowler-Nordheim theory. Since any current flowing creates an associated magnetic field, the power flow due to the emission is given by $S_{FN}=E\times H_{FN}$. 

The key point is that since the copper is a very good conductor, so to provoke a notable ohmic heating of the tip, a significant power flow through the tip is necessary. This can be calculated evaluating the $S_{FN}$ at a distance $d=h$ from the edge of the tip, where the electric field is not perturbed anymore by the shape of the tip itself. This can be formulated as the condition 
\begin{equation}
P_{RF} \ge P_{FN} \gg P_{loss}
\end{equation}
where $P_{loss}$ is the power lost for ohmic heating and $P_{FN}$ the power flow through the tip.

Considering now the relative phase of the $P_{RF}$ and of the $P_{FN}$ it is possible to derive an expression for the power emitted in a copper cavity
\begin{equation}
P_{FN} (t) = A \, E^3_0 \, sin^3 \omega t \,  \, exp \left ( \frac{-62}{\beta E_0 \, sin \, \omega t} \right )
\end{equation}
where the work function for the copper have been used is $\phi = 4.5 \, eV$, $A$ is the area of the conductor and $\omega = 2 \pi f$ is the angular velocity.

The RF power can be divided in real and imaginary part, with a phase shift of $90^\circ$. The real part is the energy propagating into the structure only and the imaginary part is the energy stored in the cavity both magnetically or electrically as happens in every resonant cavity. Since the active power flow is more efficient than the reactive one in providing power for the field emission, a weighting factor $g_c$ is introduced. Also for practice reasons since all the simulation codes work using the complex Poynting vector $\bar{S}$, the precedent reasoning can be adapted to meet the custom, using the \textit{Modified Poynting Vector}
\begin{equation}
S_c = Re\{ \bar{S} \} + g_c \, Im \{ \bar{S} \} \qquad [W/\mu m^2]
\end{equation} 

Since this quantity can be calculated in any point of a structure, this allows to identify in advance the regions which are more sensible to the breakdown process. The rule of thumb given by the current experience says that this quantity should not exceed the value of $5 W/\mu m^2$ to have a breakdown rate smaller than $1\times 10^{-6} \, bpp \, m^{-1}$ with a pulse length of 200 ns.

This quantity have been used successfully to design all the last generations of CLIC structures, including the one tested in this work.






\section[The TD26CC structure for the Main Beam of CLIC]{The TD26CC structure for the Main Beam of CLIC}

After the precedent part about the general laws that regulate the operation of the TWS, now will be presented the structure under test in this work, the CLIC\_G \textit{TD26CC}, which stays for Tapered, Damped, 26-cells active cells structure with Compact Couplers.

It's necessary to keep present in mind that in the design of the real cavities, it's almost never possible to use the general laws described in precedence because of the high complexity of the geometries. A great work of design and simulation is necessary then, and is carried out with complex numerical simulations. In this section will be presented a summa of the parameters of the cavity, the details can be found in \cite{CLIC:cdr,Grudiev:td26cc,Lunin:1333709}.









\chapter*{Abstract}

A new generation of colliders capable of reaching TeV energies is under development nowadays, and to succeed in this task it is necessary to show that the technology for such a machine is available. The Compact Linear Collider (CLIC) is a possible design option among the future lepton colliders projects. It consists of two normal-conducting linacs. To reach the required high energies within a reasonable machine length, accelerating structures with a gradient of 100 MV/m are necessary. One of the strictest requirements for such accelerating structures is a relatively low occurrence of vacuum arcs. 

CLIC prototype structures have been tested in the past, but only in absence of beam. In order to proof the feasibility of the high gradient technology for building a functional collider, it is necessary to understand the effect of the beam presence on the vacuum breakdowns. Tests of this type have never been performed so far. 

The goal of this work is to understand the effect of the beam presence in the accelerating cavity on the vacuum arcs. The setup, the experimental procedure and the results of the tests executed on a prototype accelerating cavity for the Main Beam of CLIC are described. 

The test were executed at CERN in the CLIC Test Facility 3 (CTF3), where is installed a 12 GHz X-band Test Stand (XBOX). The radio-frequency power is supplied to the cavity prototype from the XBOX, while the beam is provided from the electron linac of the CTF3, reconfigured on purpose to produce a Main Beam-like pulse. 
During the analysis the differences between the runs with and without beam will be presented. Among the different runs with beam, different effects have been detected and analysed when accelerating and decelerating the beam. 

This work offer a first glimpse on the beam effect on the breakdown rate. Further than the results presented, the experience on high gradient testing with beam has been gained. As result, in the last part some future developments are proposed to improved the experimental setup and the operation of the test stand. This will be precious in case the experiment will be replicated to complement and verify the results of this work. 
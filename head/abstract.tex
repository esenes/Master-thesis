\chapter*{Abstract}

A new generation of lepton colliders capable of reaching TeV energies is presently under development, and to succeed in this task it is necessary to show that the technology for such a machine is available. The Compact Linear Collider (CLIC) is a possible design option among the future lepton collider projects. It consists of two normal-conducting linacs. Accelerating structures with a gradient of the order of 100 MV/m are necessary to reach the required high energies within a reasonable machine length. One of the strictest requirements for such accelerating structures is a relatively low occurrence of vacuum arcs. 

CLIC prototype structures have been tested in the past, but only in absence of beam. In order to proof the feasibility of the high gradient technology for building a functional collider, it is necessary to understand the effect of the beam presence on the vacuum breakdowns. Tests of this type have never been performed previously. 

The main goal of this work is to provide a first measurement of the breakdown rate with beam in the accelerating cavity. The setup, the experimental procedure and the results of the tests executed on a prototype  cavity for the Main Beam of CLIC are described. 

The test were executed at CERN in the CLIC Test Facility 3 (CTF3), which houses a 12 GHz X-band Test Stand (XBOX). The XBOX supplies the radio-frequency power to the cavity prototype, while the beam is provided by the electron linac of the CTF3, reconfigured to produce a Main Beam-like pulse. 
A comparison between results obtained without beam and with different beam configurations will be presented, in an attempt to understand the beam effect on the breakdown rate. Moreover, future developments to improve the experimental setup and the operation of the tests stand will be proposed, based on the experience gained on high gradient testing with beam. This will be useful in case the experiment is repeated to complement and extend the results of this work.
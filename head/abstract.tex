\chapter*{Abstract}

A new generation of colliders capable of reaching TeV energies is under development nowadays, and to succede in this task is necessary to show that the technology for such machine is available. The CLIC project is one of the most advanced design among the possible lepton colliders, and is formed by two normal conducting LINACs. To reach such high energies are necessary accelerating structures carrying gradient beyond 100MV/m and one of the biggest limitations is developing accelerating structures that present a sufficient low occurrence of vacuum arcs. This is pursued both with the design and the \textit{conditioning}, which is the process of increasing the resilience to vacuum arcs of a structure using repetitive RF pulsing sessions. 

The focus of this work is on the breakdown rate testing of the TD26 type cavity with and without beam presence inside. At CERN this test has been carried out on the cavity installed in the \textit{dogleg} line in the CLIC-test-facility 3 (CTF3), and connected on the RF side to the X-band test stand 1 (Xbox1).

Other peculiar properties of the operation have been studied also, such has beam-induced RF generation into the cavity after the breakdowns, breakdown migration, ....
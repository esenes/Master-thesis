\chapter*{Italian abstract}

Una nuova generazione di acceleratori di leptoni capaci di raggiungere energie del centro di massa dell'ordine del TeV é al momento in corso di sviluppo. Per riuscire in questo intento é necessario mostrare che la tecnologia per costruire simili macchine é accessibile.

Il Compact Linear Collider (CLIC) é uno dei progetti per un acceleratori di leptoni. CLIC consiste in due linac normalmente conduttivi. Cavit\`a acceleratrici con gradienti dell'ordine dei 100 MV/m sono necessarie per raggiungere le alte energie richieste, mantenendo una lunghezza della macchina ragionevole. Uno dei parametri pi\`u stringenti per questi componenti é una occorrenza relativamente bassa di archi voltaici.

Prototipi di cavit\`a acceleratrici per CLIC sono gi\`a stati testati in passato, ma solo in assenza di fascio. Per provare la fattibilit\`a della tecnologia di accelerazione ad alto gradiente é necessario comprendere l'effetto della presenza del fascio sull'innesco degli archi voltaici.

L'obiettivo principale di questo lavoro é fornire una prima misura del rate di innesco degli archi voltaici in una cavit\`a acceleratrice in presenza di fascio. Verranno descritti il setup, le procedure sperimentali e i risultati dei test eseguiti su un prototipo di cavit\`a acceleratrice per il fascio principale di CLIC.

I test sono stati eseguiti al CERN nella CLIC Test Facility 3 (CTF3), che ospita un setup di test per radiofrequenza ad alta potenza operante a 12 GHz (detto XBOX, X-band Test Stand). L'XBOX fornisce la radiofrequenza alla cavit\`a sotto test, mentre il fascio proviene dal linac a elettroni della CTF3, configurato per produrre un fascio simile al fascio principale di CLIC. Verr\`a presenta una comparazione tra i risultati ottenuti senza fascio e con differenti configurazioni di fascio, in un primo tentativo di comprendere l'effetto del fascio sul rate di innesco degli archi voltaici. Inoltre verranno proposti sviluppi futuri per migliorare il setup sperimentale e le operazioni di test, basati sull'esperienza acquisita nei test oggetto di questo lavoro. Questo sar\`a utile, nel caso l'esperimento venisse ripetuto, per completare ed estendere i risultati di questo lavoro.
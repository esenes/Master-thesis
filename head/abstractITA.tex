\chapter*{Italian abstract}

\`E in fase di progettazione una nuova generazione di acceleratori di leptoni capaci di raggiungere energie dell'ordine del TeV. \`E tuttavia necessario mostrane la fattibilit\`a, verificando che sia accessibile la tecnologia necessaria a costruire simili macchine.

Il Compact Linear Collider (CLIC) é uno dei possibili progetti per un acceleratore lineare di leptoni, composto da due acceleratori lineari normalmente conduttivi. Per raggiungere le alte energie del centro di massa richieste, conservando allo stesso tempo una lunghezza ragionevole della macchina acceleratrice, sono necessarie cavitá di accelerazione con gradienti dell'ordine dei 100 MV/m. Uno dei parametri pi\`u stringenti per queste componenti é una occorrenza relativamente bassa di archi voltaici.

Prototipi di cavit\`a acceleratrici per CLIC sono gi\`a stati testati in passato, ma solo in assenza di fascio. Per mostrare che la tecnologia dell'accelerazione ad alto gradiente \`e affidabile, é necessario comprendere quale sia l'effetto della presenza del fascio sull'innesco degli archi voltaici. Test con la presenza del fascio non erano mai stati effettuati fino ad ora.

L'obiettivo principale di questo lavoro é fornire la prima misura del rate di innesco degli archi voltaici in una cavit\`a acceleratrice in presenza di fascio. Verranno descritti l'apparato, le procedure sperimentali e i risultati ottenuti dai test eseguiti su un prototipo di cavit\`a acceleratrice da impiegare successivamente per il fascio principale di CLIC.

I test sono stati eseguiti al CERN nella CLIC Test Facility 3 (CTF3), dove é installato un setup di test per radiofrequenza ad alta potenza operante a 12 GHz (detto XBOX, X-band Test Stand). L'XBOX fornisce la radiofrequenza alla cavit\`a in esame, mentre il fascio di elettroni impiegato proviene dal linac della CTF3, configurato per produrre un fascio simile al Main Beam di CLIC. Verr\`a presentata una comparazione tra i risultati ottenuti in assenza e con differenti configurazioni di fascio, al fine di comprendere l'effetto sul rate di innesco degli archi voltaici. Inoltre verranno proposti possibili futuri sviluppi sia per il miglioramento del setup sperimentale, sia per un perfezionamento delle procedure di test. L'esperienza acquisita durante lo sviluppo di questo lavoro e i conseguenti spunti saranno utili nel caso l'esperimento venisse ripetuto, per completare ed estendere i risultati ottenuti finora.
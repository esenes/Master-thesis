\chapter*{Acknowledgments}
\addcontentsline{toc}{chapter}{Acknowledgments}

At the end of all this work I want to aknowledge all the people that helped me to arrive so far:

Innanzi tutto voglio ringraziare i miei genitori, un grazie mia mamma perch� c'� sempre stata ogni volta che ne avevo bisogno e uno a mio pap� perch� non solo c'� sempre stato, ma per tutto quello che mi ha insegnato, perch� alla fine se me la cavo come fisico sperimentale � perch� da piccolo qualcuno mi ha portato a sporcarmi le mani con lui :)

Un grazie a mia zia Delia, non ho parole per dire quant mi abbia aiutato in questi anni e quanto mi mancher� passare anche solo a fare quattro chiacchere per un caff� nelle pause studio all'Opera.

Thanks to all the people of the CERN family met this year, going away from home is never simple, but you made this experience simply great. To Frank and Irina, for all the help, the coffe, and the good time spent together. You showed me that you can do any job without losing the smile and the laughing mood. To Theo, for all the time spent together and the precious advice that goes beyond the mere physics. To Davide for his patience and the help into and out of the CTF, and to Piotr, Jack, Tobias, Lukas, Wilfried and all the CTF3 people for the help and the advices in this my rookie year playing with accelerators. Your help was great and I really got passioned to this part of physics also because of you.

To my friends, you are too many to the acknowledged all of you, but :
A Martina, sembrava impossibile ma alla fine ce la abbiamo fatta ! 

A Davide e Andrea, 




-relatore
-cern people
	-Philip, Ola, Alvaro, Antti, Sara
-friends
	-Martina
	-D&A
	-canoisti
	-fisica
-impressing people
	-Marco
	-Enrico
	-





\blockquote{
\emph{The mediocre teacher tells. The good teacher explains. The superior teacher demonstrates. The great teacher inspires.}
\begin{flushright}
- William Arthur Ward
\end{flushright}
}%end of blockquote
Un ultimo e davvero sentito ringraziamento va al prof. Dario Mottinelli, perch� in 5 anni di liceo � riuscito a trasmettere la curiosit� e il piacere della scoperta, e solo ora mi rendo conto che per uno scienziato � la qualit� pi� importante. Tutte le volte che di fronte a un esperimento fattibile in casa non mi ha dato la risposta alle mie domande ma mi ha risposto "prova !" non ho mai capito quanto bene mi stesse facendo. Ora lo so, e gliene sar� per sempre grato. Grazie dal pi� profondo del cuore.